%-----------------------------------------
% Shen Huang
% Resume
%
% This document is covered under the Creative Commons Attribution 3.0 Unported License
%-----------------------------------------

%!TEX TS-program = xelatex
%!TEX encoding = UTF-8 Unicode

\documentclass[11pt,letterpaper,final]{moderncv}
\usepackage{fontspec}

\moderncvtheme[darkred]{casual}

% DOCUMENT LAYOUT
\usepackage[scale=0.85]{geometry}
\setlength{\hintscolumnwidth}{2cm}
\AtBeginDocument{\recomputelengths}
% FONTS
\usepackage[utf8]{inputenc}
\defaultfontfeatures{Mapping=tex-text} % converts LaTeX specials (``quotes'' --- dashes etc.) to unicode

% Remove % to set to different fonts
%\setromanfont [Ligatures={Common},Numbers={OldStyle}]{Adobe Caslon Pro}
%\setmonofont[Scale=0.8]{Monaco} 

% ---- CUSTOM AMPERSAND
\newcommand{\amper}{{\fontspec[Scale=.95]{Adobe Caslon Pro}\selectfont\itshape\&}}

% ---- MARGIN YEARS
%\newcommand{\years}[1]{\marginpar{\scriptsize #1}}


% PDF SETUP
% ---- FILL IN HERE THE DOC TITLE AND AUTHOR


% Personal Information 
\firstname{Shen}
\familyname{Huang}
\address{2156 Grove Cir W, Apt 2}{Boulder, CO 80302}
\mobile{(303) 243-2881}
\email{Shen.Huang@Colorado.edu}
\homepage{alwa.info}
\extrainfo{https://github.com/LichAmnesia}

\title{Shen Huang's Resume}

% \nopagenumbers{}
\usepackage{lastpage}%获得总页数
\usepackage{fancyhdr}
\pagestyle{fancy}
\rfoot{\thepage\ / \pageref{LastPage}}%当前页 of 总页数

\begin{document}

\maketitle
\section{Summary}
% \cventry{}{}{}{Enthusiastic in data science in and learn fast}{Want to get an intern of Software Engineer}{} 
\cvline{}{I am a software developer, passionate in top niche tools for analyzing data. I have solid knowledge of Python, Java, C/C++ and work well with Spring, Django. Well familiar with MEAN (MongoDB, Express, Angular, Node) stack. And I have experience with Hive, Hbase big data frameworks. I prefer to get an intern for software engineer.}

\section{Education}
\cventry{Aug. 2016 - }{Master of Science, Computer Science, University of Colorado Boulder}{Boulder, CO}{}{}{} % Overall GPA: 3.949}{CSC \& MA Majors GPA: 4.0
\cventry{Aug. 2012 -Jun. 2016}{Bachelor of Engineering, Computer Science, Nanjing University of Science \& Technology}{Nanjing, P.R. China}{}{}{Overall GPA: 3.6, CS major GPA: 3.6/4.0 Rank: 5\%} % Overall GPA: 3.949}{CSC \& MA Majors GPA: 4.0

\section{Work Experience} 
  % % ASC 比赛
  % \cventry{March 2016 --April 2016}{Compete in ASC Student Supercomputer Challenge}{Inspur Group Co., Ltd.}{P.R. China }{}{
  %     \begin{itemize}
  %       \item Won Prize of Excellence in the challenge.
  %       \item Designed and optimize algorithms in HPC.
  %       \item Used HPCG Test to get highest possible efficiency.
  %       \item Gave an useful optimization for the DNN program on the CPU + MIC Platform.
  %     \end{itemize}
  % }
% Clean data by AP and WiFi, write a python script of hive sql to deal with the data. Raw data is stored in Redis.
% After cleaning and deal with the data. The data should be stored in HDFS. 
  \cventry{Jan. 2016 -May. 2016}{Data Science Engineer Intern}{Future Network institute}{Nanjing, P.R. China}{}{
      \begin{itemize}
        \item Mainly used \textbf{Java} to help build a system for analyzing national network flow.
        \item Collected and analyzed the data from AP and WiFi using \textbf{Cloudera}.
        \item Used \textbf{Hive} and \textbf{Hbase} to deal with the whole log data provided by monitors. 
        \item Utilized \textbf{Spring MVC} to deal with RESTful services and \textbf{myBatis} for ORM. 
        \item Developed an API for \textbf{Redis} and wrote unit test in \textbf{JUnit}.
      \end{itemize}
  }
  % Event Create and Share
  \cventry{Sep. 2016 -}{Lead Developer}{University of Colorado Boulder}{Boulder, CO}{}{
    \begin{itemize}
      \item Built a website that people can create an event and other can participate in this event.
      \item Mainly utilized \textbf{myBatis} and \textbf{Spring} to build the back-end system.
      \item Designed front-end UI by Twitter \textbf{Bootstrap} and used \textbf{React} for building user interface.
    \end{itemize}
  }
  % OJ 开发
  % \cventry{Apr. 2014 -Jan. 2016}{Software Engineer}{NJUST}{Nanjing, P.R. China}{}{
  %     \begin{itemize}
  %       \item Maintain an online judge (njoj.org) of a non-profit organization.
  %       \item Designed a smart answering robot service system by \textbf{Python Django}.
  %       \item Utilized knowledge of Q\&A corpus to manage responses to questions. 
  %       \item Implemented \textbf{word embeddings} by employing supervised learning techniques.
  %       \item Created a service robot capable of responding to users’ tourism queries.
  %     \end{itemize}
  % }
  % 在线问答平台
  \cventry{Apr. 2014 -Jul. 2014}{Lead Developer}{Huawei Technologies Co. Ltd}{Xiamen, P.R. China}{}{
      \begin{itemize}
        \item Designed a smart answering robot service system by \textbf{Python Django}.
        \item Utilized knowledge of Q\&A corpus to manage responses to questions. 
        \item Implemented \textbf{word embeddings} by employing supervised learning techniques.
        \item Created a service robot capable of responding to users’ tourism queries.
      \end{itemize}
  }
% 集合(collection)与文档(document)
% 面向文档的存储:以 JSON 格式的文档保存数据。
\section{Student Project} 
  \cventry{Jul. 2016 -Aug. 2016}{Pandaman Generator}{University of Colorado Boulder}{Boulder, CO}{}{
    \begin{itemize}
      \item Developed a \textbf{Python} tool to generate pandaman avatar using \textbf{OpenCV}.
      \item Trained a \textbf{AdaBoost} classifier to select the face in a picture.
      \item Used a \textbf{filter} function to deal with the face and put into a pandaman avatar.
    \end{itemize}
  }
  \cventry{Sep. 2016 -Sep. 2016}{Boulder News}{University of Colorado Boulder}{Boulder, CO}{}{
    \begin{itemize}
      \item Utilized a \textbf{Python Tornado} to build a website (weather.alwa.info) to track weather in Boulder. 
      \item Wrote a spider program via \textbf{Python Requests} to get data from weather.com.
      \item Developed a system if the weather varies, it will send email and message to alert.
    \end{itemize}
  }
  \cventry{Oct. 2015 -Dec. 2015}{MusicFM website}{NJUST}{Nanjing, P.R. China}{}{
    \begin{itemize}
      \item Built an online music website (music.smilebooky.com) using \textbf{Node.js} and \textbf{Express}.
      \item Developed a \textbf{Python} sipder program to get all the music information(10 million songs) in an online music platform. Tried to analyzed and visualized all the data via \textbf{d3.js}.
      \item Performed ORM and routing assignment using \textbf{MongoDB} as database.
      \item Designed front-end UI by \textbf{Bootstrap} and \textbf{Angular}.
      \item Implemented a \textbf{Collaborative Filtering} algorithm to recommend top 5 songs.
    \end{itemize}
  }
  

\section{Research Experience} 
  % Bioinformatics 研究
  \cventry{Sep. 2015 -Jul. 2016}{Research Assistant}{Pattern Recognition and Bioinformatics Group in NJUST}{Nanjing, P.R. China}{Supervised by Dongjun Yu}{
    \begin{itemize}
      \item Tried to solve a protein problem of predicting high-dimensional structural information via little information of protein sequence.
      \item The traditional strategy for this topic used machine learning methods. Implemented these and proposed a new feature extraction method, which used 3D information of predicted protein.
      \item Tried to use \textbf{DNN model} in this problem and succeeded to get higher AUC in this problem. 
    \end{itemize}
  }
  % RM 视觉比赛
% 1.用selective search方法划分2k-45k个region;
% 2.分别对每个region提取特征,最后将提取到的特征送到k(k的取值与类别数相等) svm分类器中识别以及送到一个回归器中去调节检测框的位置;
% 3.将k个SVM分类器中得分最高的类作为分类结果,将所有得分都不高的region作为背景;
% 4.通过回归器调整之后的结果即为检测到的位置。
% http://tangxman.github.io/2015/12/10/rcnn/
% First using selective search methods to get regions. should use cnn to select features. And then train k lsvm to classify. The end is Boundingbox Regression to adjust regions. caffe-fast-rcnn
% Use tegra: make cuda, and caffe
  \cventry{Apr. 2016 -Jun. 2016}{Research Assistant}{NJUST}{Nanjing, P.R. China}{}{
    \begin{itemize}
      \item Built a machine learning framework for designing a robot, which used for detecting moving obejects via embedded device.
      \item In charge of designing unmanned vehicles with automatic target recognition and path planning function.
      \item Implemented Faster \textbf{RCNN model} for Real-time target detection.
      \item Applied the detection algorithm in \textbf{Nvidia Tegra K1} platform.
    \end{itemize}
  }

\section{HONORS \& AWARDS}
\cventry{Mar. 2016}{Outstanding Graduate}{NJUST}{Nanjing, P.R. China}{}{}
\cventry{Mar. 2016}{Prize for Excellence in Asia Student Supercomputer Challenge}{Inspur Group Co., Ltd.}{}{}{}
% \cventry{Mar. 2016}{Prize for Excellence in Asia Student Supercomputer Challenge}{Inspur Group Co., Ltd.}{}{}{Designed and optimize algorithms in HPC, and tried to optimize the \textbf{DNN} program on the \textbf{CPU + MIC} Platform}
\cventry{Nov. 2015}{Bronze Medal in ACM/ICPC - Beijing Regional}{ACM}{}{}{} % Got a rank of 62/200
\cventry{Oct. 2015}{Outstanding Students Award}{China Computer Federation}{}{}{Represented university to get this award}
\cventry{Jul. 2014}{Silver Medal in ACM/ICPC - Shanghai Inivation Contest}{ACM}{}{}{}
\cventry{Jun. 2014}{1st Prize in "Bluebridge" Cup National Software Design Contest}{Ministry of Industry and Information Technology}{Beijing, P.R. China}{}{Got a rank of 3/10K in National Final}
\cventry{Nov. 2013}{Bronze Medal in ACM/ICPC - Changsha Regional}{ACM}{}{}{}
% \cventry{Sep. 2012 -Jun. 2016}{1st Prize Scholarship}{NJUST}{Nanjing, P.R. China}{}{Won every year in university}

\section{Computer Skills}
  \cvline{Languages:}{\textbf{Python}, \textbf{Java}, \textbf{C/C++}, JavaScript, Mathematica}
  \cvline{OS:}{Linux(Ubuntu, CentOS), Windows, AWS}
  \cvline{Database:}{MySQL, MongoDb, Hbase}

\section{Extra}
  % \cvlistitem{Contributor to open source projects - see \url{https://github.com/LichAmnesia}}
  \cvlistitem{Member of the Google Developer Groups Sub (Be invited to Google I/O 2016)}
  \cvlistitem{Help Huawei Technologies Co. set about 300 interview questions in their system}
  \cvlistitem{Teaching Assistant for Course of C Programming Language for International Students}

\end{document}
